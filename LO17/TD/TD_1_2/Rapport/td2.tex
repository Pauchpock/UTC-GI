\chapter{TD2: Indexation}

Le but de ce second TD est d'utiliser le fichier XML que nous avons produit lors du TD1 afin de créer des index (des fichiers inverses). Pour cela, nous avons à notre disposition plusieurs scripts Perl :

\begin{enumerate}
  \item \textbf{segmente\_TT.pl}: permet, à partir d'un fichier XML, d'extraire tous les mots individuellement contenus entre les balises \lstinline{<titre>} ou \lstinline{<texte>}.
  \item \textbf{newcreeFiltre.pl}: permet de créer un script Perl à partir d'un fichier à une ou deux colonnes : ce script généré remplacera les mots de la première colonne par ceux de la seconde (ou par rien s'il n'y a qu'une colonne) dans un fichier donné en paramètre à ce script.
  \item \textbf{successeurs.pl} et \textbf{filtronc.pl} ; ces deux scripts s'utilisent de pair, afin de :
  \begin{enumerate}
    \item Générer la liste des succésseurs pour chaque lettre des mots d'une liste de mots.
    \item Créer, à partir des résultats obtenus avec \textbf{successeurs.pl}, un fichier à deux colonnes associant un mot à un lemme.
  \end{enumerate}
  \item \textbf{index.pl}: permet de créer à partir d'un corpus (XML) un fichier inverse sur une balise donnée en argument
  \item \textbf{indexText.pl} ; permet de créer un fichier inverse à partir d'un flux de données (entrée standard) de la forme ``mot rubrique fichier numéro''.
\end{enumerate}

\section{Choix de l'unité documentaire}

L'unité documentaire choisie est \textbf{un document = un article}. Cela nous permettra de considérer que nous avons 326 documents (N = 326, dans le calcul de \textbf{idf}), en ne prenant en compte que le titre et le texte. Selon nous, cela nous permet d'avoir un c\oe{}fficient plus précis au regard de l'ensemble des articles. Cela facilite également les traitements. En effet, considérer un bulletin nous aurait forcé à ``rassembler'' les articles, ce qui aurait rajouté de la complexité.

\section{Choix des mots à conserver}
\fakeshell

Pour toute la suite, nous avons décidé de ne traîter que des ``vrais'' mots, c'est-à-dire composés uniquement de lettres. Ainsi, dans les commandes à venir, nous exclurons \textbf{tous les mots contenant au moins un chiffre} grâce à la commande \lstinline{sed}. Effectivement, ces mots n'ont généralement aucun intérêt (par exemple ``100ème'').

\section{Obtenir les tf}

Grâce au premier script, nous sommes capable d'obtenir la liste des mots contenus dans les textes et titres de tous les documents HTML. Puis, à l'aide des commandes Unix telles que \lstinline{sort} et \lstinline{uniq} (avec l'argument \lstinline{-c}, pour compter le nombre d'occurences de chaque mot), nous obtenons pour chaque mot son \textbf{tf} (nombre d'occurences d'un mot dans un document donné).

\begin{lstlisting}
cat ../TD1/output.xml| ./segmente_TT.pl -f | sort | uniq -c | sed -e 's/^\s*//' > tf.txt
\end{lstlisting}

La commande \lstinline{sed} nous permet de supprimer les espaces inutiles en début de chaîne, qui sont automatiquement générés par les commandes.

\medskip

\noindent Exemple tronqué d'\textit{output} :
\begin{lstlisting}
5 a  67068.htm
16 à 67068.htm
2 a  67071.htm
7 à  67071.htm
8 a  67383.htm
12 à 67383.htm
\end{lstlisting}

\section{Obtenir les df}

Ensuite, pour obtenir les \textbf{df} (nombre de docuemnts dans lesquel un mot apparaît), nous avons deux possibilités: réutiliser le fichier \textbf{tf.txt} directement ou repartir du fichier XML produit lors du TD1. Voici comment faire à partir du fichier \textbf{tf.txt}:

\begin{lstlisting}
cat tf.txt | cut -d' ' -f2 | cut -f1 | sort | uniq -c | sed -e 's/^\s*//' > df.txt
\end{lstlisting}

Ici à nouveau nous retirons les espaces en début de chaine. La commande \lstinline{cut} nous permet de sélectionner uniquement certaines ``colonnes'' de notre fichier source (ici, le chaînement des deux \lstinline{cut} nous permet de récupérer uniquement la deuxième colonne, c'est-à-dire les mots).

\medskip
\noindent Exemple tronqué d'\textit{output} :
\begin{lstlisting}
244 a
326 à
1 aarhus
1 abandonnées
2 abandonner
\end{lstlisting}

\section{idf}

Ensuite, nous devons calculer les \textbf{idf} des mots. Cela implique notamment de calculer un logarithme 10. Nous avons donc créé un script Perl qui prend en argument notre fichier texte contenant les \textbf{df}. Le sortie standard de ce script est redirigée vers un fichier \textbf{idf.txt}.

\perl
\begin{lstlisting}
sub log10 {
    my $n = shift;
    return log($n)/log(10);
}

while (<>) {
    /(\d+)\s+(.*)/;
    print $2."\t".log10(326/$1)."\n";
}
\end{lstlisting}

\medskip

\noindent Exemple tronqué d'\textit{output} :
\begin{lstlisting}
a           0.12582777372921
à           0
aarhus      2.51321760006794
abandonnées 2.51321760006794
abandonner  2.21218760440396
\end{lstlisting}

\section{tf*idf}

La dernière étape avant de passer aux stop-lists est de caculer le quotient \textbf{tf*idf}. Cela se fait très facilement grâce à un script Perl utilisant un tableau associatif. Le script traite d'abord un fichier texte contenant les \textbf{idf}, en les stockant dans ce tableau. Ensuite, le script traire le fichier concernant les \textbf{tf}: c'est la que la multiplication est faite et que le résultat est affiché sur la sortie standard (que l'on va bien sûr rediriger vers un fichier).

\begin{lstlisting}
# Usage: perl % idf.txt tf.txt

@mots;
while (<>) {
    if ($ARGV =~ m/idf/) {
        /(.*?)\s+(\d+(\.\d+)?)/;
        $mots{$1} = $2;
    }
    elsif ($ARGV =~ m/tf/) {
        /(\d+)\s+(.*?)\s+(.*)/;
        $nb = $1;
        $mot = $2;
        $file = $3;
        $tfidf = $nb * ($mots{$mot});
        print $file."\t".$mot."\t".$tfidf."\n";
    }
}
\end{lstlisting}
\fakeshell

\medskip

\noindent Exemple tronqué d'\textit{output} :
\begin{lstlisting}
67068.htm	à	0
67068.htm	de	0
67068.htm	et	0
67068.htm	l	0
\end{lstlisting}


\section{Stop-liste}

Maintenant, il s'agit de générer une stoplist afin de filtrer le fichier XML généré lors du TD1. Nous avons choisi un seuil (0.79189260882435 nous a semblé correct, au regard du fichier régéré auparavant) en dessous duquel les mots seront ajoutés à la stop liste car considérés comme n'apportant rien au document (par exemple des déterminantes, des mots de liaison, etc.). Un script Perl nous permet de faire cela :

\perl
\begin{lstlisting}
# Usage: perl % tfidf.txt

@words;
while (<>) {
    /(.*?)\s(.*?)\s(\d+(\.\d+)?)/;
    if ($3 lt '0.79189260882435') {
        $words{$2} = 1;
    }
}

while (($key,$value) = each(%words)) {
    print $key."\n";
}
\end{lstlisting}

Puis, à l'aide du script fourni n°2, nous pouvons créer un script Perl qui supprimera automatiquement les mots présents dans notre stop-liste, afin de filtrer le fichier XML.

\fakeshell
\begin{lstlisting}
# Create script to remove words
./newcreeFiltre.pl stoplist.txt > removeWordsFromXML.pl && chmod +x removeWordsFromXML.pl

# Create new XML without the words from the stop list
./removeWordsFromXML.pl corpus.xml > newOutput.xml
\end{lstlisting}

\section{Création des lemmes et remplacement des mots}

Il convient ensuite des créer les lemmes. Pour cela, nous allons utiliser les scripts fournis n°3.a et 3.b, en prénant soin de toujours retirer les mots contenants des chiffres avec \lstinline{sed}:

\fakeshell
\begin{lstlisting}
# Create the successors and remove lines containing at least one number
cat newOutput.xml| ./segmente_TT.pl | sort -u | sed '/[0-9]/d' | ./successeurs_P16.pl > successeurs.txt

# Create the lemmes
./filtronc_P16.pl -v successeurs.txt successeurs.filtered.txt
\end{lstlisting}

Au regard des résultats, on constate que tout n'est pas parfait. Bien souvent, certains mots ayant la même racine n'obtiennent pas le même lemme (la taille des mots peut expliquer les différences), par exemple :

\begin{lstlisting}
mange    mange
mangent  mangent
mangeons mangeons
manger   manger
mangeur  mangeur
mangeurs mangeur
\end{lstlisting}

\medskip

La prochaine étape est simplement le remplacement des mots par leur lemmes dans le corpus XML modifié précédemment. Il nous faut pour cela créer un script Perl à l'aide du script fourni n°2 :

\begin{lstlisting}
# Create the script to replace the words with their lemmes in the XML
./newcreeFiltre.pl successeurs.filtered.txt > replaceLemmesInXML.pl && chmod +x replaceLemmesInXML.pl

# Run the script
./replaceLemmesInXML.pl ./newOutput.xml > lemmes.xml
\end{lstlisting}

\section{Création des fichiers inverses}

La dernière étape est maintenant de créer des fichiers inverses. Nous avons pour cela à disposition les scripts n°4 et 5. Nous prendrons encore le soin de retirer les mots contenants des chiffres.

\subsection{Créer un fichier inverse sur la date à l'aide du script \lstinline{index.pl}}

\fakeshell
\begin{lstlisting}
cat lemmes.xml| ./index.pl "date" > reverse.date.txt
\end{lstlisting}

Nous aurions pu choisir un autre critère que la date (comme la rubrique par exemple).

\subsection{Créer un fichier inverse sur le mot à l'aide du script \lstinline{indexText.pl}}

Il sera important ici de n'avoir que des lignes uniques pour ne pas créer de problème (un mot qui apparaît plusieurs fois dans un même fichier ne doit donner lieu qu'à une seule ligne).

\begin{lstlisting}[literate={°}{\textdegree}1]
cat ./lemmes.xml| ./segmente_TT.pl -f -n -r | sed -r '/^(\w*?)[0-9]+.*\.htm/d' | sort -u |./indexTexte.pl|sort > rev.mot.txt
\end{lstlisting}

\noindent Exemple tronqué d'\textit{output} :
\begin{lstlisting}
absorption  en direct des laboratoires  73436.htm   282 focus   72631.htm   279
abstrai     focus                       70161.htm   270 focus   72932.htm   280
abstr       au coeur des régions        69817.htm   269 focus   70420.htm   271
absurde     focus                       73684.htm   283
abusent     a lire                      72401.htm   278
\end{lstlisting}

\section{Code source}

L'intégralité du code source de ce TD2, commenté, est trouvable en annexes.

\chapter{Introduction}
\section{Résumé}
L'objectif de ce TP est d'écrire un programme permettant la gestion d'une ludothèque composée de jeux de société. Un jeu de société est décrit par une fiche contenant les informations suivantes :
\begin{itemize}
  \item Nom du jeu
  \item Genre du jeu : PLATEAU, RPG, COOPERATIF, AMBIANCE ou HASARD
  \item Le nombre de joueurs minimum
  \item Le nombre de joueurs maximum
  \item La durée moyenne d'une partie en minutes
\end{itemize}

\section{Structuration du projet}
Nous avons choisi de structuer notre projet en 5 fichiers pour une meilleure clarté, selon les deux entités principales que sont les \textbf{ludothèques} et les \textbf{jeux} :

\begin{itemize}
  \item \textbf{main.c} Le fichier source contenant le programme principal
  \item \textbf{ludotheque.h} Le fichier d'en-tête contenant les déclarations des structures et fonctions d'une ludothèque
  \item \textbf{ludotheque.c} Le fichier source contenant la définition de chaque fonction d'une ludothèque
  \item \textbf{jeu.h} Le fichier d'en-tête contenant les déclarations des structures et fonctions d'un jeu
  \item \textbf{jeu.c} Le fichier source contenant la définition de chaque fonction d'un jeu
\end{itemize}

\section{Code source}
Le code source est fourni dans l'archive ZIP contenant ce rapport.
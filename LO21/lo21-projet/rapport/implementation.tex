\chapter{Implémentation}

\section{Classes métiers}

Ayant un UML à disposition, nous avons par la suite commencé le projet en créant les classes métiers, qui sont en fait les implémentations des entités UML. Par rapport à l'UML, nous avons ajouté les constructeurs, les \textit{getters} et \textit{setters}, les attributs implicites sur l'UML (par exemple, un \lstinline{std::vector} pour les compositions), etc. Voici un extrait de la classe \lstinline{Projet} :

\code{projet.h}{C++}
\begin{lstlisting}
class Projet {
    QString nom;
    QDate dateDisponibilite;
    vector<Tache*> taches;

public:
    Projet(QString m_nom, QDate m_dispo): nom(m_nom), dateDisponibilite(m_dispo){}

    void ajouterTache(Tache *t);

    /**
     * @brief getEcheance permet de connaitre l'échéance du projet
     * @return la date d'échéance la plus éloignée, de l'ensemble des tâches du projet
     */
    QDate getEcheance() const;

    // Getters
    QString getNom() const {return nom;}
    QDate getDispo() const {return dateDisponibilite;}

    // Setters
    void setNom(const QString& m_nom) {nom = m_nom;}
    void setDispo(const QDate& m_dateDispo) {dateDisponibilite  = m_dateDispo;}

    ~Projet();
};
\end{lstlisting}

Nous avons choisi d'utiliser \textbf{trois singletons} pour les tâches, les projets et les événements (programmations) de l'agenda. Ces trois types d'entités seront respectivement contenu dans des \lstinline{TacheManage}, \lstinline{ProjetManager} et \lstinline{EvenementManager}. Cela permet d'accéder aux même objets, depuis n'importe quelle classe de l'application.

\section{Classes des interfaces graphiques}
Lors de la création des interfaces graphiques avec Qt, les classes minimales sont générées (constructeur et destructeur). Nous avons donc rajouté des méthodes pour effectuer certaines actions, comme par exemple :
\begin{itemize}
  \item Des signaux et slots pour réagir à des clics sur des éléments graphiques
  \item Des méthodes qui rafraîchissent l'ensemble de l'interface graphique
\end{itemize}

\section{Fonction \lstinline{main()}}
Notre fonction \lstinline{main()} joue un rôle très important dans l'application. Voici dans l'ordre ce qu'elle fait :
\begin{enumerate}
  \item \textbf{Instanciation des trois singletons.}
  \item \textbf{Création (si non présent) du fichier de la base SQLite dans un sous-répertoire propre à notre application, dans le répertoire de l'utilisateur.}
  \item \textbf{Connexion à la base et vérification que les tables nécessaires existent, sinon la table est \og re-initiliasée\fg{}.} Pour pouvoir faire évoluer l'application et la base facilement, il conviendrait d'avoir une table qui contient le numéro de version de l'application. Ainsi, lors des mises à jour de l'application, il serait aisé de savoir s'il faut recréer la base de données ou non.
  \item \textbf{Chargement dans les trois singletons des projets, tâches et événements contenus en base de données.}
  \item \textbf{Affichage de la fenêtre graphique.}
\end{enumerate}

\bigskip

Les différentes exceptions qui peuvent survenir sont gérées par des blocs \lstinline{try/catch} avec affichage des messages d'erreurs à l'utilisateur via une \textit{pop-up}. Par exemple :

\code{main.cpp}{C++}
\begin{lstlisting}
try {
    em->setDatabase(&sdb); // em = EvenementManager::getInstance()
    em->loadEvents();
} catch (EvenementManagerException& e) {
    fenetre.showError("Project Calendar", e.getInfo());
    app.exit(0);
    return 0;
}
\end{lstlisting}
\code{mainwindow/h}{C++}
\begin{lstlisting}
void showError(QString titre, QString description) {
    QMessageBox::warning(this, titre, description);
}
\end{lstlisting}
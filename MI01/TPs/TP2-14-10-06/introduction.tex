\chapter{Introduction}

Le but de ce TP est d'étudier la modélisation en VHDL séquentiel de différents circuits. Nous commencerons par réaliser un simple compteur 2 bits synchrone. Puis nous réaliserons le détecteur de code que nous avons étudié en TD, en détectant les fausses entrées ou non.

\medskip

En plus de programmer le \textbf{FPGA}, nous devrons également réaliser les simulations pour étudier les minimisations et optimisations de synthèse effectuées par le logiciel. En effet, si le code VHDL écrit n'est pas optimal, le logiciel effectuera des optimisations de manière automatique.
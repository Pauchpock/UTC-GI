\chapter{Conclusion générale}

\section{OpenGL ou Unity}

La plus grosse problématique du projet aura été le choix entre Unity et OpenGL. Nous étions initalement partis sur du développement en OpenGL mais heureusement, en début d'année 2015, Unity a rendu son moteur totalement gratuit, nous permettant ainsi de pouvoir l'utiliser. Nous nous en sommes rendus compte peu de temps après le début du projet et avons ainsi changé de technologie. Nous avons tout de même eu le temps d'essayer OpenGL et il s'était avéré que le développement aurait été bien trop dur et presque impossible. Effectivement, nous ne connaissions pas cette technologie et la contrainte de temps (un semestre) ne nous aurait pas permis de l'appréhender.

\section{Problèmes}

Un autre problème aura été l'utilisation du SDK de Google. Au départ nous ne savions pas comment faire déplacer le joueur, nous étions donc partis sur l'utilisation d'une \textit{library} tierce, Dive. Or il s'est avéré que cette \textit{library} souffrait d'un bug au niveau du gyroscope. Nous avons finalalement réussi à utiliser uniquement le SDK de Google et faire avancer notre joueur.

\section{Jeu final et objectifs}

Au final, un seul de nos objectifs complémentaires a été atteint : rajouter une ambiance sonore (une musique de fond). Nous sommes néanmoins très satisfaits du résultat visuel et du jeu de manière générale. Avoir plus de temps nous aurait probablement permis d'atteindre d'autres objectifs.
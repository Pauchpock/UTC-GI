\documentclass[11pt,a4paper,oneside,french,svgnames]{report}
\usepackage[utf8]{inputenc} % force the use of utf8
\usepackage[T1]{fontenc} % font encoding, allows accents
\usepackage[papersize={21cm,29.7cm},top=2.0cm,bottom=2.0cm, inner=2.0cm, outer=2.0cm]{geometry} % page formatting
\usepackage[french]{babel} % translate everything in the desired language: table of contents, etc. 'english' can be replaced with 'francais'
\usepackage{graphicx} % images management
\usepackage{wrapfig} % floating images
\usepackage{array} % allow arrays
\usepackage{fancyhdr} % headers/footers management (overrides empty, plain and headings)
\usepackage{listings} % code insertion (MUST BE WRITTEN AFTER BABEL)
\usepackage[nottoc,numbib]{tocbibind} % bib in toc
\usepackage{pdfpages} % include PDF documents
\usepackage{enumitem} % for /setlist
\usepackage{color,soul} % add some colors and hightlight
\usepackage{xcolor} % more colors
\usepackage[hyphens]{url} % auto break lines in URL
\usepackage[hidelinks,  colorlinks  = true, % no borders, colors enabled
                        anchorcolor = blue,
                        linkcolor   = black, % links in table of contents
                        urlcolor    = blue,
                        citecolor   = blue]{hyperref}

\sethlcolor{cyan} % package soul
\newcommand{\file}[1]{\hl{\emph{#1}}} % highlight a file URI

%%%%%%%%%%%%%%%%%%%%%%%%%%%%%%%%%%%%%%%%%%%%%%%%%%%%%%%% LISTINGS %%%%%%%%%%%%%%%%%%%%%%%%%%%%%%%%%%%%%%%%%%%%%%%%%%%%%%%%
\definecolor{comment}{rgb}{0.12, 0.38, 0.18 } % adjusted, in Eclipse: {0.25, 0.42, 0.30 } = #3F6A4D
\definecolor{keyword}{rgb}{0.37, 0.08, 0.25}  % #5F1441
\definecolor{string}{rgb}{0.06, 0.10, 0.98} % #101AF9
\setlength{\headheight}{30pt}
% "define" Scala
\lstdefinelanguage{scala}{
  morekeywords={abstract,case,catch,class,def,%
    do,else,extends,false,final,finally,%
    for,if,implicit,import,match,mixin,%
    new,null,object,override,package,%
    private,protected,requires,return,sealed,%
    super,this,throw,trait,true,try,%
    type,val,var,while,with,yield},
  otherkeywords={=>,<-,<\%,<:,>:,\#,@},
  sensitive=true,
  morecomment=[l]{//},
  morecomment=[n]{/*}{*/},
  morestring=[b]",
  morestring=[b]',
  morestring=[b]"""
}

\lstset{
  columns=flexible, %prevent extra spaces
  rulecolor=\color{black!50},
  backgroundcolor = \color{blue!10},
  numbers=none, % line numbering
  showspaces=false,
  showtabs=false,
  breaklines=true,
  showstringspaces=false,
  breakatwhitespace=false,
  commentstyle=\color{comment},
  keywordstyle=\color{keyword},
  stringstyle=\color{string},
  basicstyle=\ttfamily,
  extendedchars=true,
  emph=[2]{In},
  emphstyle=[2]\color{black!70},
  morecomment=[l][\color{blue}]{Out},
  frame=single,
  frameround=tttt,
  framerule=0.3pt,
  framesep=4pt,
  belowcaptionskip=2.1pt,
  literate={à}{{\`a} }1 {â}{{\^a}}1 %                         letter a
           {À}{{\`A}}1 {Â}{{\^A}}1 %                         letter A
           {ç}{{\c{c}}}1 %                                   letter c
           {Ç}{{\c{C}}}1 %                                   letter C
           {é}{{\'e}}1 {è}{{\`e}}1 {ê}{{\^e}}1 {ë}{{\"e}}1 % letter e
           {É}{{\'E}}1 {È}{{\`E}}1 {Ê}{{\^E}}1 {Ë}{{\"E}}1 % letter E
           {î}{{\^i}}1 {ï}{{\"i}}1 %                         letter i
           {Î}{{\^I}}1 {Ï}{{\"I}}1 %                         letter I
           {ô}{{\^o}}1 %                                     letter o
           {Ô}{{\^O}}1 %                                     letter O
           {œ}{{\oe}}1 %                                     letter oe
           {Œ}{{\OE}}1 %                                     letter OE
           {ù}{{\`u}}1 {û}{{\^u}}1 {ü}{{\"u}}1 %             letter u
           {Ù}{{\`U}}1 {Û}{{\^U}}1 {Ü}{{\"U}}1 %             letter U
  % above is a hack to force UTF8 compatibility (only for french)
}

\newcommand{\java}{\lstset{
  language=java,
  title={{\setlength{\fboxsep}{1pt}\fcolorbox{orange}{yellow!20}{\sffamily\scriptsize
              \textcolor{gray!10}{\_}Java\textcolor{gray!10}{\_}}}}
  }
}

\newcommand{\textcode}[1]{\lstset{
  language=,
  title={{\setlength{\fboxsep}{1pt}\fcolorbox{orange}{yellow!20}{\sffamily\scriptsize
              \textcolor{gray!10}{\_}{#1}\textcolor{gray!10}{\_}}}}
  }
}

\newcommand{\scala}{\lstset{
  language=scala,
  title={{\setlength{\fboxsep}{1pt}\fcolorbox{orange}{yellow!20}{\sffamily\scriptsize
              \textcolor{gray!10}{\_}Scala\textcolor{gray!10}{\_}}}}
  }
}

\newcommand{\cpp}{\lstset{
  language=C++,
  title={{\setlength{\fboxsep}{1pt}\fcolorbox{orange}{yellow!20}{\sffamily\scriptsize
              \textcolor{gray!10}{\_}C++\textcolor{gray!10}{\_}}}}
  }
}

%%%%%%%%%%%%%%%%%%%%%%%%%%%%%%%%%%%%%%%%%%%%%%%%%%%%%%%%%%%%%%%%%%%%%%%%%%%%%%%%%%%%%%%%%%%%%%%%%%%%%%%%%%%%%%%%%%%%%%%%%%%

%\parindent=20pt
\fancypagestyle{plain}{
    % Headers
    \fancyhead[R]{Rapport SY31}
    \fancyhead[L]{Mohamed Baaziz\\Yi Luo\\Romain Pellerin}

    % Footers
    \renewcommand{\footrulewidth}{0.1pt}
    \fancyfoot[C]{\today}
    \fancyfoot[LE]{\ifnum\thepage>0 \thepage \fi}
    \fancyfoot[RO]{\ifnum\thepage>0 \thepage \fi}
}

\fancypagestyle{empty}{%
    \renewcommand{\headrulewidth}{0pt} % No sub line
    \fancyhead{} % Empty the header

    \renewcommand{\footrulewidth}{0pt}
    \fancyfoot{}
} 

\setlist[itemize,2]{label={$\bullet$}} % use bullets for nested itemize

% First page
\newcommand{\presentation}[1]{\vspace{0.3cm}\large{\textbf{#1}}\vspace{0.3cm}\\}
\newcommand{\presentationLarge}[1]{\vspace{0.3cm}\LARGE{\textbf{#1}}\vspace{0.3cm}\\}

% Overrides chapter (numbered and no-numbered) headings: remove space, display only the title
\makeatletter
  \def\@makechapterhead#1{%
  \vspace*{0\p@}% avant 50
  {\parindent \z@ \raggedright \normalfont
    %\ifnum \c@secnumdepth >\m@ne
    %    \huge\bfseries \@chapapp\space \thechapter
    %    \par\nobreak
    %    \vskip 20\p@
    %\fi
    \interlinepenalty\@M
    \Huge \bfseries \thechapter\quad #1   
    \vskip 40\p@
  }}
  \def\@makeschapterhead#1{%
  \vspace*{0\p@}% before 50
  {\parindent \z@ \raggedright
    \normalfont
    \interlinepenalty\@M
    \Huge \bfseries  #1\par\nobreak
    \vskip 40\p@
  }}
\makeatother

\newcommand{\ignore}[1]{} % inline comments

\pagenumbering{arabic}
%\addtocounter{page}{-7} % page numbering starts at 1 + (-7)
\pagestyle{plain} % uses fancy

\title{Intership report}
\author{Romain PELLERIN}
\date\today

%\setcounter{tocdepth}{4}

\begin{document}

\chapter{Programmation du composant}
Afin de générer le composant RTMaps, nous avons dû, à partir d’un squelette fourni, écrire les fonctions C++ nécessaires au calcul de l’odométrie du robot.

\medskip

\noindent Trois classes nous étaient fournies :
\begin{itemize}
  \item OdometryComponent
  \item PositionViewer
  \item Wifibot
\end{itemize}
Il nous a fallu remplir le corps des fonctions de \textbf{odometryComponent.cpp} et de \textbf{wifibot.cpp}.

\section{odometry-component.cpp}
Cette classe représente le composant RTMaps. Elle gère toute la durée de vie du composant : naissance, fonctionnement et mort (arrêt du composant) ainsi que sa réinitialisation.
Voici le code que nous avons complété, commenté :
\cpp
\begin{lstlisting}
//...
const double NB_TICKS_PER_WHEEL_TURN = 2448; // = 2Pi
const double WIFIBOT_ENTRAX_IN_METER = 0.3;
const double WIFIBOT_RADIUS_IN_METER = 0.0625;
const double TWOPI = 2 * M_PI;
//...

////////////////////////////////////////////////////////////////
// RTMaps - Birth
////////////////////////////////////////////////////////////////
void
OdometryComponent::Birth()
{
  // Appel du constructeur avec les paramètres requis
  mWifibot=Wifibot(WIFIBOT_ENTRAX_IN_METER,WIFIBOT_RADIUS_IN_METER);
  // On stocke le timestamp (temps) actuel en microsecondes
  mPreviousTimestamp = MAPS::CurrentTime();
}

////////////////////////////////////////////////////////////////
// RTMaps - Core
////////////////////////////////////////////////////////////////
void
OdometryComponent::Core() 
{
  //...

  // On stocke le temps écoulé en secondes depuis le dernier appel de la fonction
  double elapsedTime = (MAPS::CurrentTime() - mPreviousTimestamp) / 1000000;
  // Vitesse actuelles des roues gauche et droite en tick/s
  double vLeft  = static_cast<double>(input->Integer32(0));
  double vRight = static_cast<double>(input->Integer32(1));
  // Conversion de tick/s vers radian/s
  vLeft = (vLeft*TWOPI)/NB_TICKS_PER_WHEEL_TURN;
  vRight = (vRight*TWOPI)/NB_TICKS_PER_WHEEL_TURN;
  // Appel de la méthode updateOdometry de l'objet mWifibot pour mettre à jour son odométrie en fonction du temps écoulé, de la vitesse des roues gauche et droite en radian/s
  mWifibot.updateOdometry(elapsedTime,vLeft,vRight);

  //...

  // On stocke le timestamp actuel en nmicrosecondes
  mPreviousTimestamp = MAPS::CurrentTime();

}

void
OdometryComponent::reset()
{
  mWifibot.resetOdometry(); // On réinitialise l'odométrie du robot (toutes les valeurs à zéro)
}

//...
\end{lstlisting}

\section{wifibot.cpp}
Cette classe représente le Wifibot. Elle permet de modifier les paramètres du robot (l’entrax, le rayon des roues ou l’odométrie). Elle permet également d’obtenir des informations sur le robot (la vitesse linéaire et la vitesse angulaire). 

\medskip

Pour la plupart des fonctions, nous avons réutilisé les formules fournies dans l'énoncé du TP. Voici le code que nous avons complété, expliqué en commentaires :
\begin{lstlisting}
//...

// Constructeur qui prend comme paramètres l'entraxe et le rayon des roues
Wifibot::Wifibot(double entrax, double wheelRadius)
{
  mEntrax = entrax;
  mWheelRadius = wheelRadius;
}

// Mise à jour de l'odométrie du robot. On fournit en entrée le temps écoulé depuis la dernière mise à jour de l'odométrie (en secondes), ainsi que la vitesse des roues gauche et droite (en mètres/s)
void
Wifibot::updateOdometry(double dt, double left, double right)
{
  // La position (en x ou en y) est mise à jour de la manière suivante : on ajoute à l'ancienne position la vitesse linéaire actuelle, multipliée par le cosinus de l'orientation actuelle. La vitesse linéaire est obtenue grâce à une méthode de la même classe (voir plus bas).
  mPosition.x  += dt * getLinearSpeed(left, right)*cos(mPosition.th);
  mPosition.y  += dt * getLinearSpeed(left, right)*sin(mPosition.th);

  // L'orientation est mise à jour de la façon suivante : on ajoute à l'ancienne orientation la valeur négative du rayon des roues, multipliée par la position x additionnée à y, divisées par l'entraxe.
  mPosition.th += -mWheelRadius*(mPosition.x+mPosition.y)/mEntrax;
}

// Remise à zéro de toutes les valeurs de l'odométrie
void
Wifibot::resetOdometry()
{
  mPosition.x = 0.0;
  mPosition.y = 0.0;
  mPosition.th = 0.0;
}

// Modifie l'entraxe
void
Wifibot::setEntrax(double entrax)
{
  mEntrax = entrax;
}

// Modifie le rayon des roues
void
Wifibot::setWheelRadius(double radius)
{
  mWheelRadius = radius;
}

// Retourne la vitesse linéaire en fonction de la vitesse des roues de gauche et de droite
double
Wifibot::getLinearSpeed(double left, double right) const
{
  // Vitesse linéaire = rayon multiplié par la vitesse des roues gauche + la vitesse des roues droite, le tout divisé par 2
  return mWheelRadius*( left + right ) / 2;
}

// Retourne la vitesse angulaire en fonction de la vitesse des roues de gauche et de droite
double
Wifibot::getAngularSpeed(double left, double right) const
{
  // Vitesse angulaire = rayon multiplié par la vitesse des roues gauche - la vitesse des roues droites, le tout divisé par l'entraxe
  return  -mWheelRadius*(left - right) / mEntrax;
}
\end{lstlisting}
\end{document}
